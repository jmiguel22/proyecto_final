\title{ESTUDIO DE CASO SOBRE LA RELACIÓN ENTRE REDES SOCIALES Y ESCUELA EN LOS PROCESOS DE APRENDIZAJE}
\author{
        Juan Miguel Mart\'inez \\
                Departamento de Educaci\'on\\
        Universidad de Granada\\
}
\date{\today}

\documentclass[12pt]{article}
\usepackage[utf8]{inputenc}
\usepackage[spanish]{babel}
\usepackage{hyperref}
\begin{document}
\maketitle

\begin{abstract}
Este trabajo tiene como objeto de estudio comprender la influencia de las interacciones entre los escenarios escolares y virtuales en los procesos de aprendizaje. As\'i pues, se ha llevado a cabo un estudio de caso de un alumno de ESO (Educaci\'on Secundaria Obligatoria) de 15 años de edad. Como herramientas de recogida de informaci\'on hemos utilizado las entrevistas en profundidad. Los resultados muestran el gran potencial que tienen las redes sociales para desarrollar el aprendizaje, la identidad y el capital social en los adolescentes.\\\\
Enlace al repositorio: \url{https://github.com/jmiguel22/proyecto_final.git}\\\\
\textbf{Palabras clave:} Aprendizaje, Redes sociales

\end{abstract}

\section{Introducci\'on}
En la era de la informaci\'on en la que vivimos, el uso de las redes sociales se ha convertido en un factor determinante a la hora de fomentar la comunicaci\'on, la participaci\'on y la formaci\'on entre los j\'ovenes. En este sentido, la irrupci\'on de esta nueva forma de obtener, compartir y gestionar la informaci\'on, plantea nuevos escenarios de aprendizaje dentro y fuera de las instituciones educativas. 

\section{Estado del arte}
Surgen nuevos espacios virtuales que se reservan directamente para el aprendizaje. Sloep y Berlanga \cite{sloep_learning_2011}, presentan el concepto de redes de aprendizaje como \''entornos de aprendizaje en l\'inea que ayudan a los participantes a desarrollar sus competencias colaborando y compartiendo informaci\'on\'' (p.56). As\'i, las redes sociales tambi\'en tienen un componente educativo que ofrece una serie de ventajas como: el aprendizaje colaborativo y cooperativo, la construcci\'on de grupos, y la creaci\'on de conocimiento de forma colectiva \cite{ortega_espacios_2008}. En este sentido, aparecen los MOOC como redes virtuales destinadas a la enseñanza de contenidos, esto es, entornos virtuales de aprendizaje favorecidos por la tecnolog\'ia \cite{bartolome-pina_are_2015}.

\section{Im\'agenes y tablas}

\section{F\'ormulas}

\bibliographystyle{abbrv}
\bibliography{biblio}

\end{document}
This is never printed
