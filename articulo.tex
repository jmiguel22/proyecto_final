\title{ESTUDIO DE CASO SOBRE LA RELACIÓN ENTRE REDES SOCIALES Y ESCUELA EN LOS PROCESOS DE APRENDIZAJE}
\author{
        Juan Miguel Mart\'inez \\
                Departamento de Educaci\'on\\
        Universidad de Granada\\
}
\date{\today}

\documentclass[12pt]{article}
\usepackage[utf8]{inputenc}
\usepackage[spanish]{babel}
\usepackage{hyperref}
\begin{document}
\maketitle

\begin{abstract}
Este trabajo tiene como objeto de estudio comprender la influencia de las interacciones entre los escenarios escolares y virtuales en los procesos de aprendizaje. As\'i pues, se ha llevado a cabo un estudio de caso de un alumno de ESO (Educaci\'on Secundaria Obligatoria) de 15 años de edad. Como herramientas de recogida de informaci\'on hemos utilizado las entrevistas en profundidad. Los resultados muestran el gran potencial que tienen las redes sociales para desarrollar el aprendizaje, la identidad y el capital social en los adolescentes.\\\\
Enlace al repositorio: \url{https://github.com/jmiguel22/proyecto_final.git}\\\\
\textbf{Palabras clave:} Aprendizaje, Redes sociales

\end{abstract}

\section{Introducci\'on}
En la era de la informaci\'on en la que vivimos, el uso de las redes sociales se ha convertido en un factor determinante a la hora de fomentar la comunicaci\'on, la participaci\'on y la formaci\'on entre los j\'ovenes. En este sentido, la irrupci\'on de esta nueva forma de obtener, compartir y gestionar la informaci\'on, plantea nuevos escenarios de aprendizaje dentro y fuera de las instituciones educativas. 

\section{Estado del arte}
A much longer \LaTeXe{} example was written by Gil~\cite{Gil:02}.

\section{Results}\label{results}
In this section we describe the results.

\section{Conclusions}\label{conclusions}
We worked hard, and achieved very little.

\bibliographystyle{abbrv}
\bibliography{biblio}

\end{document}
This is never printed
